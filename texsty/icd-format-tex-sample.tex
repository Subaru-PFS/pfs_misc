\documentclass[a4paper,notitlepage]{article}
\usepackage{icd-format}
\drafttrue
\title{Usage and Sample for PFS ICD LaTeX style}
\author{PFS Project Office}
\begin{document}

\ICDID{00000}
\ICDREV{000}
\ICDCATEGORY{ALL}
\ICDITSIDS{}  % comma separated ID numbers, eg. 1,2,3

\ICDApproval{
    XXXXXXXX (XXXX--XX--XX) \\
    XXXXXXXX (XXXX--XX--XX) \\
    XXXXXXXX (XXXX--XX--XX)
}
\ICDSuperApproval{XXXXXXXX (XXXX--XX--XX)}
\ICDChangeRecord{Rev.001 / First Release / 2012--07--20}
\ICDReference{(none); if any put ICD ID here}
\ICDAttachment{(none); if any put 22char file id of document server}

\icdhead

\section{Abstract}

This document covers usage and sample to use icd--format.tex style file. 
Also you can copy source for this sample as a template. 


\section{\LaTeX\ header}

You need to put some commands at the beginning of your \LaTeX\ source file. 

\begin{verbatim}
\documentclass[a4paper,notitlepage]{article}
\usepackage{icd-format}
\drafttrue
\title{Usage and Sample for PFS ICD LaTeX style}
\author{PFS Project Office}
\begin{document}

\ICDID{00000}
\ICDREV{000}
\ICDCATEGORY{ALL}
\ICDITSIDS{}  % comma separated ID numbers, eg. 1,2,3
\ICDApproval{
    XXXXXXXX (XXXX--XX--XX) \\
    XXXXXXXX (XXXX--XX--XX) \\
    XXXXXXXX (XXXX--XX--XX)
}
\ICDSuperApproval{XXXXXXXX (XXXX--XX--XX)}
\ICDChangeRecord{Rev.001 / First Release / 2012--07--20}
\ICDReference{(none); if any put ICD ID here}
\ICDAttachment{(none); if any put 22char file id of document server}

\icdhead
\end{verbatim}

Each command means: 

\begin{description}
  \item[drafttrue] If the ICD document is draft, insert this line. 
  \item[ICDID] ID for the ICD, if it is a draft for the first revision, input 
    '00000'.
  \item[ICDREV] Revision number for the ICD document, keep this as the latest 
    issued revision number (or '000' if new one). 
    Including {\tt \textbackslash drafttrue} will add '+' after this revision number.
  \item[ICDITSIDS] Related ITS issue \#, with comma separated, like 
    '1,2,3'.
  \item[ICDApproval] Approval record for the last revision of the ICD.
    If draft, keep this blank. Use format of 'Name (YYYY--MM--DD)'. 
    You can use '\textbackslash\textbackslash' for multi-line.
  \item[ICDSuperApproval] Super approval record for the last revision of the 
    ICD.
    If draft, keep this blank. Use format of 'Name (YYYY--MM--DD)'. 
    You can use '\textbackslash\textbackslash' for multi-line.
  \item[ICDChangeRecord] Change record from the first revision. At least, you 
    need to include revision number, release short description, and release 
    date. 
    You can use '\textbackslash\textbackslash' for multi-line.
  \item[ICDReference] Related ICD issue ID, which any related discussions for 
    the latest release was held. 
    If draft, include the current on-going ITS issue ID. 
  \item[ICDAttachment] Related document ID in the document server (22char, 
    not document ID in numberic). 
    Any file related to the (latest revision of) ICD should be included. 
\end{description}

Also, you need to input {\tt \textbackslash icdhead} just before your content, 
and you don't need to include {\tt \textbackslash maketitle} or {\tt 
\textbackslash tableofcontents}. 


\section{Building document}

You need {\bf pdflatex} to build document. 
Also place {\tt icd--format.sty} and {\tt pfsstyle.sty} files into the same 
directory of \LaTeX\ source file or {\tt TEXMF} directory 
(like {\tt \$\{HOME\}/texmf/tex/latex/}). 


\end{document}

